%%%%%%%%%%%%%%%%%%%%%%%%%%%%%%%%%%%%%%%%%
% "ModernCV" CV and Cover Letter
% LaTeX Template
% Version 1.1 (9/12/12)
%
% This template has been downloaded from:
% http://www.LaTeXTemplates.com
%
% Original author:
% Xavier Danaux (xdanaux@gmail.com)
% 
% Updated by: Armin Hadzic
%
% License:
% CC BY-NC-SA 3.0 (http://creativecommons.org/licenses/by-nc-sa/3.0/)
%
% Important note:
% This template requires the moderncv.cls and .sty files to be in the same
% directory as this .tex file. These files provide the resume style and themes
% used for structuring the document.
%
%%%%%%%%%%%%%%%%%%%%%%%%%%%%%%%%%%%%%%%%%

%----------------------------------------------------------------------------------------
%   PACKAGES AND OTHER DOCUMENT CONFIGURATIONS
%----------------------------------------------------------------------------------------

\documentclass[11pt,a4paper,sans]{moderncv} % Font sizes: 10, 11, or 12; paper sizes: a4paper, letterpaper, a5paper, legalpaper, executivepaper or landscape; font families: sans or roman

\moderncvstyle{banking} % CV theme - options include: 'casual' (default), 'classic', 'oldstyle' and 'banking'
\moderncvcolor{blue} % CV color - options include: 'blue' (default), 'orange', 'green', 'red', 'purple', 'grey' and 'black'

\usepackage[scale=0.88, 
			lmargin=0.95cm, 
			rmargin=0.95cm, 
			footnotesep=1.5cm
			]{geometry} % Reduce document margins

%bibliograph content
\usepackage[minnames=8,
			maxnames=10,
			backend=bibtex,
            style=numeric-comp,
            sorting=ydnt,
            backref=false,
            defernumbers=true
            ]{biblatex}
\usepackage{booktabs}
\usepackage[unicode,
			colorlinks=true,
			urlcolor=blue,
			]{hyperref}
\usepackage{academicons}
\usepackage{fontspec}

\providecommand{\Accessible}[2]{#2}  % header-safe: ignore ActualText

% Publication bib files
\addbibresource{publications/conferences.bib}
\addbibresource{publications/journals.bib}
\addbibresource{publications/workshops.bib}
%\addbibresource[label=under_review]{under_review}
%\addbibresource[label=abstracts]{abstracts}


%Title spacing
\makeatletter
\patchcmd{\makehead}{%search
    \ifthenelse{\equal{\@title}{}}{}{\titlestyle{~|~\@title}}\\%
    }{%replace
    \ifthenelse{\equal{\@title}{}}{}{\titlestyle{~|~\@title}}\\[0.25cm]%
  }{%success
  }{%failure
  }
\makeatother


% To support more-parsable icons
\setmainfont{TeX Gyre Heros} % or Source Sans 3, Inter, etc.

% \moderncvicons{awesome}      % tell moderncv to use Font Awesome icons
% \usepackage{fontawesome5}    % make sure FA5 is available

% Remove bullet from contact information, replace with 2 spaces
\renewcommand*{\makeheaddetailssymbol}{\space|\space}

% Proper google scholar icon
\renewcommand*{\googlescholarsocialsymbol}{%
  \Accessible{[Papers]}{\aiGoogleScholar}%
  \textnormal{\kern0.25em}%
}

% Tell hyperref what to write in PDF strings (bookmarks/metadata)
\pdfstringdefDisableCommands{%
  % Map BOTH the icon macro and the moderncv symbol macro:
  \def\aiGoogleScholar{Google Scholar}%
  \renewcommand*{\googlescholarsocialsymbol}{Google Scholar}%
}



%----------------------------------------------------------------------------------------
%   NAME AND CONTACT INFORMATION SECTION
%----------------------------------------------------------------------------------------

\firstname{Armin} % Your first name
\familyname{Hadzic} % Your last name

% All information in this block is optional, comment out any lines you don't need
%\title{Curriculum  LaTeX Error: Missing \begin{document}.Vitae}
%\address{}
%\mobile{(\#\#\#) \#\#\#-\#\#\#\#}
\email{first.last@outlook.com}
%\email{armin.hadzic.ai@gmail.com}

% The first argument is the url for the clickable link, the second argument is the url displayed in the template - this allows special characters to be displayed such as the tilde in this example
\homepage{www.arminhadzic.com}
\social[linkedin][www.linkedin.com/in/armin-hadzic]{LinkedIn}
\social[googlescholar][scholar.google.com/citations?user=DlDme3IAAAAJ]{Papers}
\social[github][github.com/arminHadzic]{Code}

%----------------------------------------------------------------------------------------


\begin{document}

\makecvtitle % Print the CV title
\vspace*{-1.5cm}

%----------------------------------------------------------------------------------------
%   RESEARCH INTERESTS SECTION
%----------------------------------------------------------------------------------------
% \section{}
% \cvitem{}{I design and deliver scalable AI systems that measurably improve performance and push the state-of-the-art. My research interests touch on multimodal fusion, computer vision (e.g., video generation, 3D vision), and real-world outcomes.}

%----------------------------------------------------------------------------------------
%   PROFESSIONAL EXPERIENCE SECTION
%----------------------------------------------------------------------------------------

\section{Professional Experience}
\cventry{Fairfax, VA}{\textsc{DZYNE Technologies Inc.}}{Senior AI Research Scientist}{2021-Present}{}
{
\begin{itemize}
	\item Developed multimodal fusion models 5TB of videos for a multi-org initiative (Rutgers, WashU, UKy); boosting F1 from $52\%$ to $75\%$ and securing $\$500k$ in follow-on for $\$2.3M$ program (WACV~\cite{greenwell2024watch}).
	\item Designed a multimodal transformer with contextual representations for traffic modeling; exceeding the prior state-of-the-art (SOTA) by 7\% (ECCV~\cite{ProbTraffic}).
	\item Automated multi-resolution imagery processing and dataset construction ($>100k$ samples), resulting in a $\$1M$ extension after achieving $78\%$ segmentation F1.
	\item Co-led 9 proposal efforts across an 11-person team; securing $\$3M$ to support AI R\&D.
	\item Developed a generative approach for segmentation super‑resolution that reached $88\%$ accuracy (WACV~\cite{Low2High}).
\end{itemize}
}

%% TODO: AR/VR MLDE, GPSD
%------------------------------------------------

\cventry{Laurel, MD}{\textsc{Johns Hopkins University Applied Physics Lab}}{AI Research Scientist}{2020-2021}{}
{
\begin{itemize} 
	\item Trained and deployed vision models at scale (Docker/Dask/Slurm) for the Climate Trace initiative; delivered $CO_2$ emissions estimates at global scale via Microsoft Planetary Computer (CVPRW~\cite{mukherjee2021towards}).
	\item Developed adversarial de-biasing techniques to improve AI fairness in medical data/imaging applications by $20\%$ (MICCAI~\cite{yuan2023edgemixup}, Neural Computation~\cite{paul2022tara}).
	\item Optimized multi-agent swarm control; reduced cooperative capture time by $25s$ for multiple rewards (Array~\cite{hadzic2022100218}, ISEC~\cite{buckley2021interdisciplinary}).
	\item Designed a CUDA kernel for hierarchical video object detection; increased classification specificity under tight latency.
\end{itemize}
}

%------------------------------------------------

\cventry{Lexington, KY}{\textsc{University of Kentucky Computer Vision Lab}}{Research Assistant}{2017-2020}{}
{
\begin{itemize}
	\item Built a multimodal dataset (LiDAR, imagery) and architecture that outperformed the prior SOTA by $13\%$ (CVPRW~\cite{RasterNetpaper}).
	\item Designed NLP models to forecast company performance based on financial text filings, and benchmarked on evaluation pipelines.
	\item Led the development of a safety-focused, perception-driven, navigation system (WACV~\cite{WACVpaper}).
	\item Collaborated with CDC, UNHCR, and IOM on population estimation for resource allocation, reducing error to $7\%$ (IGARSS~\cite{DCApaper}).
\end{itemize}
}

%------------------------------------------------

\cventry{Lexington, KY}{\textsc{Belcan Engineering Group Inc.}}{Software Development Engineer}{2015-2018}{}
{
\begin{itemize}
	\item Automated jet engine diagnostics in C/C++, saving $\$100k$ by developing a diagnostic and fault resolution system.
	\item Streamlined a legacy C++/Make cross-platform building system, reducing development and compilation time by $60\%$.
\end{itemize}
}

%------------------------------------------------

\cventry{2012-2014}{Internships}{}{}{}
{
\begin{itemize}
	\item Designed a routing algorithm for $20+$ Automated Guided Vehicles, reducing scrap and transportation costs by $\$57k$/year.
\end{itemize}
}

%----------------------------------------------------------------------------------------
%   TECHNICAL SKILLS SECTION
%----------------------------------------------------------------------------------------

\section{Technical Skills}

\begin{tabular}{l@{\qquad}|>{\hspace{0.5pc}}l@{\qquad}} % The final bracket specifies the number of columns in the table along with left and right borders which are specified using vertical bars (|); each column can be left, right or center-justified using l, r or c. To specify a precise width, use p{width}, e.g. p{5cm}.  The qquads specify verticle line dividers. hspace is used to specify the space between text and verticle dividers.

%\toprule % Optional Top horizontal line
Languages 					& Python, C/C++, Java, Verilog, \LaTeX, Shell \\ \hline % row 1
AI/ML 						& PyTorch, Keras, BoTorch, Multimodal Transformers, LLMs, Reinforcement Learning \\ \hline % row 2
Computer Vision 			& Generative AI, Segmentation, 3D Vision, Localization, Pose, Depth, Remote Sensing \\ \hline % row 3
Infrastructure 				& Data Processing, Distributed Training, Slurm, Docker, Optimization, AWS, A\&B Testing \\ % row 4
\end{tabular}

%----------------------------------------------------------------------------------------
%   EDUCATION SECTION
%----------------------------------------------------------------------------------------

\section{Education}
\cventry{Advisor: Nathan Jacobs}{Master of Science in Computer Science}{University of Kentucky}{2018-2020}{\textit{GPA -- 4.0, Outstanding MS Student Award}}{}
\cventry{Magna Cum Laude}{Bachelor of Science in Computer Engineering}{University of Kentucky}{2016}{\textit{GPA -- 3.8}}{}
\cventry{Magna Cum Laude}{Bachelor of Science in Electrical Engineering}{University of Kentucky}{2009-2013}{\textit{GPA -- 3.8}}{} 
% Arguments not required can be left empty

%----------------------------------------------------------------------------------------
%   SERVICE SECTION
%----------------------------------------------------------------------------------------

\section{Service \& Recognition}
\cvitem{}
{
\begin{itemize}
	\item Best Paper ISEC 2022 and CVPRW EARTHVISION 2021; Outstanding Reviewer CVPR 2024 \& 2025.
	\item Reviewer 2022-2025: CVPR, ECCV, ICCV, ICLR, NeurIPS, WACV, and CVPRW EARTHVISION.
\end{itemize}
}

\pagebreak
%----------------------------------------------------------------------------------------
%   PUBLICATIONS SKILLS SECTION
%----------------------------------------------------------------------------------------

\printbibheading [title={Publications}]
\printbibliography[
    title={Conference Papers},
    keyword={conferences},
    heading=subbibliography, 
    resetnumbers=false]
\printbibliography[
    title={Journal Articles},
    keyword={journals},
    heading=subbibliography, 
    resetnumbers=false]
\printbibliography[
    title={Workshop Papers},
    keyword={workshops}, 
    heading=subbibliography, 
    resetnumbers=false]

\end{document}
