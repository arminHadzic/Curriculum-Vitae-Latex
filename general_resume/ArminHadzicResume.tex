%%%%%%%%%%%%%%%%%%%%%%%%%%%%%%%%%%%%%%%%%
% "ModernCV" CV and Cover Letter
% LaTeX Template
% Version 1.1 (9/12/12)
%
% This template has been downloaded from:
% http://www.LaTeXTemplates.com
%
% Original author:
% Xavier Danaux (xdanaux@gmail.com)
% 
% Updated by: Armin Hadzic
%
% License:
% CC BY-NC-SA 3.0 (http://creativecommons.org/licenses/by-nc-sa/3.0/)
%
% Important note:
% This template requires the moderncv.cls and .sty files to be in the same
% directory as this .tex file. These files provide the resume style and themes
% used for structuring the document.
%
%%%%%%%%%%%%%%%%%%%%%%%%%%%%%%%%%%%%%%%%%

%----------------------------------------------------------------------------------------
%   PACKAGES AND OTHER DOCUMENT CONFIGURATIONS
%----------------------------------------------------------------------------------------

\documentclass[11pt,a4paper,sans]{moderncv} % Font sizes: 10, 11, or 12; paper sizes: a4paper, letterpaper, a5paper, legalpaper, executivepaper or landscape; font families: sans or roman

\moderncvstyle{banking} % CV theme - options include: 'casual' (default), 'classic', 'oldstyle' and 'banking'
\moderncvcolor{blue} % CV color - options include: 'blue' (default), 'orange', 'green', 'red', 'purple', 'grey' and 'black'
\usepackage{xcolor} 

\usepackage[scale=0.88, 
			lmargin=0.95cm, 
			rmargin=0.95cm, 
			footnotesep=1.5cm
			]{geometry} % Reduce document margins

%bibliograph content
\usepackage[minnames=8,
			maxnames=10,
			backend=bibtex,
            style=authoryear,
            natbib=true, 
            style=numeric-comp,
            sorting=ydnt,
            backref=false,
            defernumbers=true
            ]{biblatex}
\usepackage{booktabs}
\usepackage[unicode,
			colorlinks=true,
			urlcolor=blue,
			]{hyperref}
 

% Publication bib files
\addbibresource{publications/conferences}
\addbibresource{publications/journals}
\addbibresource{publications/workshops}
%\addbibresource[label=under_review]{under_review}
%\addbibresource[label=abstracts]{abstracts}


%Title spacing
\makeatletter
\patchcmd{\makehead}{%search
    \ifthenelse{\equal{\@title}{}}{}{\titlestyle{~|~\@title}}\\%
    }{%replace
    \ifthenelse{\equal{\@title}{}}{}{\titlestyle{~|~\@title}}\\[0.25cm]%
  }{%success
  }{%failure
  }
\makeatother

% Remove bullet from contact information, replace with 2 spaces
\renewcommand*{\makeheaddetailssymbol}{~~}

%----------------------------------------------------------------------------------------
%   NAME AND CONTACT INFORMATION SECTION
%----------------------------------------------------------------------------------------

\firstname{Armin} % Your first name
\familyname{Hadzic} % Your last name

% All information in this block is optional, comment out any lines you don't need
%\title{Curriculum  LaTeX Error: Missing \begin{document}.Vitae}
%\address{}
%\mobile{(\#\#\#) \#\#\#-\#\#\#\#}
\email{firstlast at outlook.com}

% The first argument is the url for the clickable link, the second argument is the url displayed in the template - this allows special characters to be displayed such as the tilde in this example
\homepage{www.arminhadzic.com}
\social[linkedin][www.linkedin.com/in/armin-hadzic]{LI}
\social[googlescholar][scholar.google.com/citations?user=DlDme3IAAAAJ&hl=en]{Publications}

%----------------------------------------------------------------------------------------


\begin{document}

\makecvtitle % Print the CV title
\vspace*{-1.5cm}

%----------------------------------------------------------------------------------------
%   EDUCATION SECTION
%----------------------------------------------------------------------------------------

\section{Education}
\cventry{Advisor: Nathan Jacobs}{Master of Science in Computer Science}{University of Kentucky}{2018-2020}{\textit{GPA -- 4.0, Outstanding MS Student Award}}{}
\cventry{Magna Cum Laude}{Bachelor of Science in Computer Engineering}{University of Kentucky}{2016}{\textit{GPA -- 3.8}}{}
\cventry{Magna Cum Laude}{Bachelor of Science in Electrical Engineering}{University of Kentucky}{2009-2013}{\textit{GPA -- 3.8}}{} 
% Arguments not required can be left empty

%----------------------------------------------------------------------------------------
%   PROFESSIONAL EXPERIENCE SECTION
%----------------------------------------------------------------------------------------

\section{Professional Experience}
\cventry{Fairfax, VA}{\textsc{DZYNE Technologies Inc.}}{AI Research Scientist}{2021-Present}{}
{
\begin{itemize}
	\item Designed and trained a feature extractor and hierarchical weighted sampler for a multimodal fusion model on a \textbf{5TB AWS dataset} across 5 organizations—boosting the \textbf{F1 score from $\mathbf{52\%}$ to $\mathbf{75\%}$} and \textbf{securing} $\mathbf{\$500k}$ in funding for a $\$2.3M$ program.
	\item Developed scalable tools for processing and constructing $\mathbf{100k}$\textbf{+ sample image datasets}, resulting in a $\mathbf{\$1M}$ \textbf{contract extension} after achieving $\mathbf{78\%}$ \textbf{F1 score} in segmentation.
	\item Led/co-wrote 9 proposals, securing \textbf{$\mathbf{\$3M}$ in funding} to support AI R\&D for a team of 11 researchers and engineers.
	\item \textbf{Achieved $\mathbf{88\%}$ accuracy} in land cover semantic segmentation by developing a GAN-based super-resolution label synthesis method, with strong applicability to street-level imagery problems.
	\item Engineered a multimodal transformer (PyTorch, HuggingFace) with contextual representations, surpassing the prior state-of-the-art method by reducing traffic modeling error to $\mathbf{7\%}$ on a $\mathbf{12k}$ \textbf{sample} dataset.
\end{itemize}
}

%% TODO: AR/VR MLDE, GPSD
%------------------------------------------------

\cventry{Laurel, MD}{\textsc{Johns Hopkins University Applied Physics Lab}}{AI Research Scientist}{2020-2021}{}
{
\begin{itemize} 
	\item Modeled greenhouse gas emissions for the Climate TRACE initiative, trained with Slurm and deployed with Docker \& Dask on the Microsoft Planetary Computer, achieving a $\mathbf{39kg/100m^2}$ \textbf{error rate} across the USA and running at global scale.
	\item Optimized multi-agent swarm controllers via BoTorch, reducing cooperative capture time by $\mathbf{25s}$ for multiple rewards.
	\item Developed adversarial de-biasing techniques that \textbf{enhanced AI fairness by $\mathbf{20\%}$} in medical imaging and data applications, enabling broader applicability across large populations.
	\item Developed a state-scale satellite image approach, collapsing the displaced community population estimation error down to $7\%$.
\end{itemize}
}

%------------------------------------------------

\cventry{Lexington, KY}{\textsc{University of Kentucky Computer Vision Lab}}{Research Assistant}{2017-2020}{}
{
\begin{itemize}
	\item Developed a multimodal (point cloud/imagery) road \textbf{dataset ($\mathbf{1M}$+ segments)} using distributed computing (Slurm). This improved free-flow speed estimation by $\mathbf{13\%}$--outperforming the prior state-of-the-art with a novel multimodal fusion architecture.
	\item Designed a PyTorch NLP model to analyze SEC reports, attaining $\mathbf{41\%}$ \textbf{tercile accuracy} in predicting financial returns. Leveraged NLTK and SpaCy for efficient text processing and tokenization.
\end{itemize}
}

%------------------------------------------------

\cventry{Lexington, KY}{\textsc{Belcan Engineering Group Inc.}}{Software Development Engineer}{2015-2018}{}
{
\begin{itemize}
	\item Automated jet engine diagnostics in C/C++, \textbf{saving $\mathbf{\$100,000}$} by developing a diagnostic and fault resolution system.
	\item Streamlined a legacy C++/Make cross-platform building system, reducing development and compilation time by $60\%$.
\end{itemize}
}

%------------------------------------------------

\cventry{2012-2014}{\textsc{Internships}}{}{}{}
{
\begin{itemize}
	\item Designed a routing algorithm for over 20 Automated Guided Vehicles, reducing scrap and transportation costs by $\mathbf{\$57k}$\textbf{/year}.
\end{itemize}
}

%----------------------------------------------------------------------------------------
%   TECHNICAL SKILLS SECTION
%----------------------------------------------------------------------------------------

\section{Technical Skills}

\begin{tabular}{l@{\qquad}|>{\hspace{0.5pc}}l@{\qquad}} % The final bracket specifies the number of columns in the table along with left and right borders which are specified using vertical bars (|); each column can be left, right or center-justified using l, r or c. To specify a precise width, use p{width}, e.g. p{5cm}.  The qquads specify verticle line dividers. hspace is used to specify the space between text and verticle dividers.

%\toprule % Optional Top horizontal line
Languages 					& \textbf{Python}, C/C++, Java, Verilog, \LaTeX, Shell \\ \hline % row 1
AI/ML 						& \textbf{PyTorch}, Keras, Tensorflow, \textbf{Multimodal Transformers}, LLMs, Reinforcement Learning \\ \hline % row 2
Computer Vision 			& Generative AI, Segmentation, 3D Vision, Localization, Pose, Depth, Remote Sensing \\ \hline % row 3
Infrastructure 				& Data Processing, Distributed Training, Slurm, Docker, Optimization, AWS, Evaluation \\ % row 4
\end{tabular}

%----------------------------------------------------------------------------------------
%   SERVICE SECTION
%----------------------------------------------------------------------------------------

\section{Service \& Recognition}
\cvitem{}
{
\begin{itemize}
	\item Best Paper ISEC 2022 and CVPRW EARTHVISION 2021; Outstanding Reviewer CVPR 2024 \& 2025.
	\item Reviewer 2022-2025: CVPR, ECCV, ICCV, ICLR, NeurIPS, WACV, and CVPRW EARTHVISION.
\end{itemize}
}

\end{document}
