%%%%%%%%%%%%%%%%%%%%%%%%%%%%%%%%%%%%%%%%%
% "ModernCV" CV and Cover Letter
% LaTeX Template
% Version 1.1 (9/12/12)
%
% This template has been downloaded from:
% http://www.LaTeXTemplates.com
%
% Original author:
% Xavier Danaux (xdanaux@gmail.com)
% 
% Updated by: Armin Hadzic
%
% License:
% CC BY-NC-SA 3.0 (http://creativecommons.org/licenses/by-nc-sa/3.0/)
%
% Important note:
% This template requires the moderncv.cls and .sty files to be in the same
% directory as this .tex file. These files provide the resume style and themes
% used for structuring the document.
%
%%%%%%%%%%%%%%%%%%%%%%%%%%%%%%%%%%%%%%%%%

%----------------------------------------------------------------------------------------
%   PACKAGES AND OTHER DOCUMENT CONFIGURATIONS
%----------------------------------------------------------------------------------------

\documentclass[11pt,a4paper,sans]{moderncv} % Font sizes: 10, 11, or 12; paper sizes: a4paper, letterpaper, a5paper, legalpaper, executivepaper or landscape; font families: sans or roman

\moderncvstyle{banking} % CV theme - options include: 'casual' (default), 'classic', 'oldstyle' and 'banking'
\moderncvcolor{burgundy} % CV color - options include: 'blue' (default), 'orange', 'green', 'red', 'purple', 'grey' and 'black'

\usepackage[scale=0.88, lmargin=0.95cm, rmargin=0.95cm, footnotesep=1.5cm]{geometry} % Reduce document margins

%bibliograph content
\usepackage[minnames=8,
				maxnames=10,
				backend=bibtex,
                style=authoryear,
                natbib=true, 
                style=numeric-comp,
                sorting=ydnt               
                ]{biblatex}
\usepackage{booktabs}

% Publication subheadings
\defbibheading{journal}{Journal Publications}
\defbibheading{conference}{Publications}
\defbibheading{workshop}{Workshop Publications}
\defbibheading{abstracts}{Abstracts}

% Publication bib files
\addbibresource[label=conference]{conferences}
\addbibresource[label=journal]{journals}
\addbibresource[label=workshop]{workshops}
\addbibresource[label=under_review]{under_review}

%Title spacing
\makeatletter
\patchcmd{\makehead}{%search
    \ifthenelse{\equal{\@title}{}}{}{\titlestyle{~|~\@title}}\\%
    }{%replace
    \ifthenelse{\equal{\@title}{}}{}{\titlestyle{~|~\@title}}\\[0.25cm]%
  }{%success
  }{%failure
  }
\makeatother


%----------------------------------------------------------------------------------------
%   NAME AND CONTACT INFORMATION SECTION
%----------------------------------------------------------------------------------------

\firstname{Armin} % Your first name
\familyname{Hadzic} % Your last name

% All information in this block is optional, comment out any lines you don't need
%\title{Curriculum Vitae}
%\address{}
%\mobile{(\#\#\#) \#\#\#-\#\#\#\#}
\email{firstname lastname at outlook dot com}
%Can add webpage when it is up.
\homepage{www.arminhadzic.com}
% The first argument is the url for the clickable link, the second argument is the url displayed in the template - this allows special characters to be displayed such as the tilde in this example
\social[linkedin][www.linkedin.com/in/armin-hadzic]{armin-hadzic}

%----------------------------------------------------------------------------------------


\begin{document}

\makecvtitle % Print the CV title
\vspace*{-1.5cm}

%----------------------------------------------------------------------------------------
%   RESEARCH INTERESTS SECTION
%----------------------------------------------------------------------------------------
\section{Research Interests}

\cvitem{}{Developing unsupervised/self-supervised methods to address challenges in novel class discovery and latent information representation, especially across multiple modalities (e.g., imagery, audio, and point clouds). More generally, I am interested in \textbf{deep learning, computer vision, reinforcement learning, natural language,} and \textbf{artificial intelligence}.}

%----------------------------------------------------------------------------------------
%   EDUCATION SECTION
%----------------------------------------------------------------------------------------

\section{Education}
\cventry{Advisor: Nathan Jacobs}{Master of Science in Computer Science}{University of Kentucky}{2018-2020}{\textit{GPA -- 4.0}}{Thesis: Estimating Free-Flow Speed with LiDAR and Overhead Imagery}
\cventry{}{Bachelor of Science in Computer Engineering}{University of Kentucky}{2016}{\textit{GPA -- 3.8}}{Graduated Magna Cum Laude}
\cventry{}{Bachelor of Science in Electrical Engineering}{University of Kentucky}{2009-2013}{\textit{GPA -- 3.8}}{Graduated Magna Cum Laude, Minor in Computer Science}  % Arguments not required can be left empty
\cvitem{}{Computer Science Outstanding MS Student 2020.}
\cvitem{}{Dean's List Fall 2010 to Spring 2013.}


%----------------------------------------------------------------------------------------
%   WORK EXPERIENCE SECTION
%----------------------------------------------------------------------------------------

\section{Professional Experience}
\subsection{Research}
\cventry{Fairfax, VA}{\textsc{DZYNE Technologies Inc.}}{Computer Vision Research Scientist}{2021-Present}{}
{
\begin{itemize} 
	\item Researching and developing computer vision deep learning methods for overhead and ground-level imagery.
	\item Designed methods for addressing resolution mismatch between imagery and annotations in supervised learning.
	\item Led or contributed to writing 9 proposals (SBIR, STTR, BAA), winning one for a 5-year \$3M program.
\end{itemize}
}
\cventry{Laurel, MD}{\textsc{Johns Hopkins University Applied Physics Laboratory}}{Computer Vision Researcher}{2020-2021}{}
{
\begin{itemize} 
	\item Designed and implemented deep learning methods for applied research in computer vision, remote sensing, medical imaging, and neuroscience.
	\item Developed models robust to bias in classification of skin diseases, reducing skin tone bias by 7\% while maintaining 85\% accuracy.
	\item Integrated geospatial products into artificial neural networks for high resolution building damage classification, structure localization, and green house gas regression.  
\end{itemize}
}
\cventry{Lexington, KY}{\textsc{UK Computer Vision Lab}}{Research Assistant}{2018-2020}{}
{
\begin{itemize} 
	\item Advised by Professor Nathan Jacobs.
	\item Designed multi-modal neural networks to leverage point clouds and satellite imagery to estimate free-flow speeds of roads.
	\item Developed Natural Language Processing (NLP) temporal convolutional and attention-based neural network models to estimate firm economic performance using public SEC text reports. 
\end{itemize}
}
\cventry{Laurel, MD}{\textsc{Johns Hopkins University Applied Physics Laboratory}}{Machine Perception Intern}{2019}{}
{
\begin{itemize} 
	\item Advised by Ryan Mukherjee and Dr. Gordon Christie.
	\item Regressed population of displaced communities for disaster relief efforts, utilizing overhead imagery and deep neural networks.
\end{itemize}
}
\cventry{Lexington, KY}{\textsc{UK Computer Vision Lab}}{Volunteer Machine Learning Research Assistant}{2017-2018}{}
{
\begin{itemize} 
	\item Automated the US Road Assessment Program (usRAP) road safety assessment using a deep convolutional neural network to directly estimate roadway safety based on street-level panorama images, reducing evaluation time to milliseconds per image.
	\item Integrated the roadway safety estimator into a GPS vehicle routing system to enhance navigation with the capability to identify a balanced, safe and fast, driving route.
\end{itemize}
}

%------------------------------------------------

\subsection{Industry}

\cventry{Lexington, KY}{\textsc{Belcan Engineering Group Inc.}}{Software Development Engineer}{2017-2018}{}
{
	\begin{itemize}
		\item Developed, maintained, and tested a jet engine diagnostic and fault resolution system, saving over \$100,000 by automating engine maintenance diagnostics.
		\item Integrated and streamlined a legacy cross-platform build system with modern development tools, mitigating build errors and reducing development time.
	\end{itemize}
}

%------------------------------------------------

\cventry{Lexington, KY}{\textsc{Belcan Engineering Group Inc.}}{Embedded Software Engineer}{2016-2017}{}
{
\begin{itemize}
	\item Streamlined the user interface and reduced diagnostic time of jet engines by identifying, isolating, and purging Onboard Maintenance System inefficiencies and defects.
\end{itemize}
}

%------------------------------------------------

\cventry{Lexington, KY}{\textsc{Belcan Engineering Group Inc.}}{Software Test Engineer}{2015-2016}{}
{
\begin{itemize}
	\item Designed and implemented Control and Diagnostic System Verification and Validation Tests for 4 P\&W Turbofan Jet Engines. 
	\item Discovered mission critical control logic, software, and documentation defects through root-cause analysis, informal testing, regression testing and system testing; leading to best in class, safe, and high performance jet engines.
\end{itemize}
}

%------------------------------------------------

\cventry{Lexington, KY}{\textsc{Tempur Sealy International Inc}}{Software Engineering Co-op}{2013-2014}{}
{
\begin{itemize}
	\item Pioneered and developed a GUI and 3D topography mapping application to visually analyze large datapoint datasets, generating streamlined product testing, seamless user experience, and refined product quality.
\end{itemize}
}
%------------------------------------------------

\cventry{Florence, KY}{\textsc{Johnson Controls Inc}}{Software Engineering Intern}{2012}{}
{
\begin{itemize}
	\item Designed and implemented a software algorithm for streamlined Automated Guided Vehicle (AGV) routing, saving \$57,000 per year in scrap reduction and transportation costs.
\end{itemize}
}

%----------------------------------------------------------------------------------------
%   PUBLICATIONS SKILLS SECTION
%----------------------------------------------------------------------------------------
\section{Publications}

\begin{refsection}[conference]
	\nocite{*}
	\printbibliography[title={Conferences},heading=subbibliography]
\end{refsection}
\begin{refsection}[journal]
	\nocite{*}
	\printbibliography[title={Journals},heading=subbibliography]
\end{refsection}
\begin{refsection}[workshop]
	\nocite{*}
	\printbibliography[title={Workshops},heading=subbibliography]
\end{refsection}
% \begin{refsection}[under_review]
% 	\nocite{*}
% 	\printbibliography[title={Under Review},heading=subbibliography]
% \end{refsection}

%\nocite{WACVpaper}
%\printbibliography[title=Publications]

%----------------------------------------------------------------------------------------
%   TECHNICAL SKILLS SECTION
%----------------------------------------------------------------------------------------

\section{Technical skills}

\begin{tabular}{l@{\qquad}|>{\hspace{0.5pc}}l@{\qquad}} % The final bracket specifies the number of columns in the table along with left and right borders which are specified using vertical bars (|); each column can be left, right or center-justified using l, r or c. To specify a precise width, use p{width}, e.g. p{5cm}.  The qquads specify verticle line dividers. hspace is used to specify the space between text and verticle dividers.

%\toprule % Top horizontal line

Programming Languages & Python, C/C++, Verilog, Java, \LaTeX, Assembly, Make \\ \hline % Content row 1
Libraries & PyTorch, Scikit-Learn, Keras, Tensorflow \\ \hline % Content row 2
Operating Systems & Unix/Linux, Windows, OSX, Android \\ \hline % Content row 3
Development Environments & Linux Toolchain, PyCharm, Visual Studio, Android Studio, Xilinx \\ %Content row 4

\bottomrule % Bottom horizontal line
\end{tabular}

%----------------------------------------------------------------------------------------
%   INTERESTS SECTION
%----------------------------------------------------------------------------------------

\section{Service}
\cvitem{}{
\subsection{Technical Committee}
\begin{itemize}
\item 2021 University of Maryland Honor's program Gemstone thesis defense.
\end{itemize}
\subsection{Program Committee/Reviewing}
\begin{itemize}
\item ICCV: 2023
\item ECCV: 2022
\item CVPR: 2023
\item ICLR: 2024
\item NeurIPS: 2023
\item WACV Round 1 and 2: 2023, 2024
\item CVPR Workshop EARTHVISION: 2022, 2023
\end{itemize}
}

\end{document}
